\documentclass[11pt,a4paper,oneside]{report}

\input{include/packages}

\newcommand{\vikszerzoVezeteknev}{Fintha}
\newcommand{\vikszerzoKeresztnev}{Dénes Flórián}

\newcommand{\vikkonzulensAMegszolitas}{Dr.~}
\newcommand{\vikkonzulensAVezeteknev}{Bergmann}
\newcommand{\vikkonzulensAKeresztnev}{Gábor}

\newcommand{\vikkonzulensBMegszolitas}{}
\newcommand{\vikkonzulensBVezeteknev}{}
\newcommand{\vikkonzulensBKeresztnev}{}

\newcommand{\vikkonzulensCMegszolitas}{}
\newcommand{\vikkonzulensCVezeteknev}{}
\newcommand{\vikkonzulensCKeresztnev}{}

\newcommand{\vikcim}{Performance analysis of a language tooling}
\newcommand{\viktanszek}{\bmemit}
\newcommand{\vikdoktipus}{\bsc}
\newcommand{\vikmunkatipusat}{szakdolgozatot}

\input{include/tdk-variables}
\newcommand{\szerzoMeta}{\vikszerzoVezeteknev{} \vikszerzoKeresztnev}
\input{include/thesis-en}
\input{include/preamble}

\begin{document}

\pagenumbering{gobble}
\selectthesislanguage
\include{include/titlepage}
\tableofcontents\cleardoublepage
\include{include/declaration}

% --- ABSTRACT --------------------------------------------------------------- %

\pagenumbering{roman}
\setcounter{page}{1}

\selecthungarian
\chapter*{Kivonat}\addcontentsline{toc}{chapter}{Kivonat}
A szakdolgozat egy modellek validációjával foglalkozó eszköz teljesítményével
kapcsolatos problémáinak nyomozásáról és orvoslásáról szól.

A modellek validációja önmagában egy erőforrásigényes feladat, így nem
meglepő, hogy egy ilyen eszközben találunk javítani valót a teljesítmény terén.
A vizsgált eszköz a VIATRA Query, melyben mintákat definiálhatunk, és
illeszthetünk modellekre az esetleges hibák felderítésére, egy kifejezetten erre
szolgáló nyelv segítségével (VIATRA Query Language, VQL). A közelmúltban több
felhasználónak is feltűnt, hogy a VQL szerkesztő indokolatlanul lassan reagál
a lekérdezések szerkesztésére, mely megakadályozza a gördülékeny munkamenetet.

A dolgozatban nagy vonalakban ismertetem a VIATRA Query-t, mint szoftvert,
valamint mutatok néhány egyszerű példát a használatára minták írásán keresztül.

A szoftver bemutatása után egy ismerten lassú példafájl használatával bemutatom
a probléma forrásának felderítését egy profilozó szoftver segítségével. Emellett
részletekbe menően ismertetem a releváns részek működését, valamint kidolgozok
egy potenciális javítást is.

Végül miután elkészült a javítás, utolsó lépésként bemutatom, hogyan lehet a
változtatásokat elküldeni a VIATRA fejlesztőinek, hogy a szoftver következő
kiadása már tartalmazhassa őket.
\vfill

\selectenglish
\chapter*{Abstract}\addcontentsline{toc}{chapter}{Abstract}
This thesis is about investigating and solving the performance issues of a tool,
which validates models.

Model validation by itself is a resource-intensive task. As such, it's not
surprising, that we can find things to improve in such a tool.
The tool we inspect is VIATRA Query, in which we can defined patterns, and match
them on models to check them for potential errors. This is done in a
domain-specific language (VIATRA Query Language, VQL). In the past, multiple
users have noticed, that the VQL editor reacts in an unreasonably slow manner to
changes in the VQL editor, which obstructs a smooth workflow.

In the thesis, I'll broadly show the VIATRA Query software, and show the
basic usage of it through some simple patterns.

After showing the tool itself, I'll present the investigation of the problem by
using a file, that is known to cause slowdowns, and a profiler software. Along
with this, I'll show the details about how that part of the software works, and
work out a potential fix for it, too.

Finally, after the my fix is in a working state, I'll show how to send changes
to the developers of VIATRA, so my fix could be included in its next release.
\vfill

% ---------------------------------------------------------------------------- %

\cleardoublepage
\selectthesislanguage
\newcounter{romanPage}
\setcounter{romanPage}{\value{page}}
\stepcounter{romanPage}

\pagenumbering{arabic}

% --- CONTENT ---------------------------------------------------------------- %

\chapter{Introduction}
\section{Problem description}
\section{Work environment}
\subsection{Eclipse Platform for VIATRA development}
\subsection{YourKit Java Profiler}
\subsection{The VIATRA source code}

\chapter{Finding the cause of slow editors}
\section{Profiling the query language editor}
\subsection{The profiling workflow}
\subsection{Dynamic rebuild}
\subsection{The functions taking the most time}
\section{Type inference in VIATRA Query}
\subsection{Re-generating metamodels}
\subsection{Possible causes of the slowdowns}
\section{Conclusion}

\chapter{Optimizing query metamodel generation}
\section{Solution ideas}
\subsection{TODO: Possible solution 1}
\subsection{TODO: Possible solution 2}
\section{Choosing a viable solution}
\section{Implementing the chosen solution}
\subsection{TODO: Solution step 1}
\subsection{TODO: Solution step 2}
\subsection{TODO: Solution step 3}
\section{Comparing the optimized state with the original one}
\subsection{A first look}
\subsection{Original samples}
\subsection{Fixed samples}

\chapter{Applying changes to the main code repository}
\section{Writing documentation and tests}
\subsection{The asciidoc format}
\section{Opening a bug tracker ticket}
\section{Sending the changes to version control}
\section{Asking for code review}
\section{Integrating changes into the next VIATRA version}

% ---------------------------------------------------------------------------- %

%\listoffigures\addcontentsline{toc}{chapter}{\listfigurename}
%\listoftables\addcontentsline{toc}{chapter}{\listtablename}

\addcontentsline{toc}{chapter}{\bibname}
\nocite{*} % FIXME remove
\bibliography{bib/mybib}

\label{page:last}
\end{document}
