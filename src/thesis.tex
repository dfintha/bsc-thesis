\documentclass[11pt,a4paper,oneside]{report}

% thanks to http://tex.stackexchange.com/a/47579/71109
\usepackage{pdfpages}
\usepackage{ifxetex}
\usepackage{ifluatex}
\newif\ifxetexorluatex % a new conditional starts as false
\ifnum 0\ifxetex 1\fi\ifluatex 1\fi>0
   \xetexorluatextrue
\fi

\ifxetexorluatex
  \usepackage{fontspec}
\else
  \usepackage[T1]{fontenc}
  \usepackage[utf8]{inputenc}
  \usepackage[lighttt]{lmodern}
  \ttfamily\DeclareFontShape{T1}{lmtt}{m}{it}{<->sub*lmtt/m/sl}{}
\fi

\usepackage[english,magyar]{babel} % Alapértelmezés szerint utoljára definiált nyelv lesz aktív, de később külön beállítjuk az aktív nyelvet.

\usepackage{emptypage} % omit page number on empty pages

%\usepackage{cmap}
\usepackage{amsfonts,amsmath,amssymb} % Mathematical symbols.
%\usepackage[ruled,boxed,resetcount,linesnumbered]{algorithm2e} % For pseudocodes. % beware: this is not compatible with LuaLaTeX, see http://tex.stackexchange.com/questions/34814/lualatex-and-algorithm2e
\usepackage{booktabs} % For publication quality tables for LaTeX
\usepackage{graphicx}

%\usepackage{fancyhdr}
%\usepackage{lastpage}

\usepackage{anysize}
%\usepackage{sectsty}
\usepackage{setspace} % For setting line spacing

\usepackage[unicode]{hyperref} % For hyperlinks in the generated document.
\usepackage{xcolor}
\usepackage{listings} % For source code snippets.

\usepackage[amsmath,thmmarks]{ntheorem} % Theorem-like environments.

\usepackage[hang]{caption}

\singlespacing

\newcommand{\selecthungarian}{
	\selectlanguage{magyar}
	\setlength{\parindent}{2em}
	\setlength{\parskip}{0.5em}
	\frenchspacing
}

\newcommand{\selectenglish}{
	\selectlanguage{english}
	\setlength{\parindent}{0em}
	\setlength{\parskip}{0.5em}
	\nonfrenchspacing
	\renewcommand{\figureautorefname}{Figure}
	\renewcommand{\tableautorefname}{Table}
	\renewcommand{\partautorefname}{Part}
	\renewcommand{\chapterautorefname}{Chapter}
	\renewcommand{\sectionautorefname}{Section}
	\renewcommand{\subsectionautorefname}{Section}
	\renewcommand{\subsubsectionautorefname}{Section}
}

\usepackage[numbers]{natbib}
\usepackage{xspace}


\newcommand{\vikszerzoVezeteknev}{Fintha}
\newcommand{\vikszerzoKeresztnev}{Dénes Flórián}

\newcommand{\vikkonzulensAMegszolitas}{Dr.~}
\newcommand{\vikkonzulensAVezeteknev}{Bergmann}
\newcommand{\vikkonzulensAKeresztnev}{Gábor}

\newcommand{\vikkonzulensBMegszolitas}{}
\newcommand{\vikkonzulensBVezeteknev}{}
\newcommand{\vikkonzulensBKeresztnev}{}

\newcommand{\vikkonzulensCMegszolitas}{}
\newcommand{\vikkonzulensCVezeteknev}{}
\newcommand{\vikkonzulensCKeresztnev}{}

\newcommand{\vikcim}{Performance analysis of a language tooling}
\newcommand{\viktanszek}{\bmemit}
\newcommand{\vikdoktipus}{\bsc}
\newcommand{\vikmunkatipusat}{szakdolgozatot}

\input{include/tdk-variables}
\newcommand{\szerzoMeta}{\vikszerzoVezeteknev{} \vikszerzoKeresztnev}
\input{include/thesis-en}
%--------------------------------------------------------------------------------------
% Page layout setup
%--------------------------------------------------------------------------------------
% we need to redefine the pagestyle plain
% another possibility is to use the body of this command without \fancypagestyle
% and use \pagestyle{fancy} but in that case the special pages
% (like the ToC, the References, and the Chapter pages)remain in plane style

\pagestyle{plain}
\marginsize{35mm}{25mm}{15mm}{15mm}

\setcounter{tocdepth}{3}
%\sectionfont{\large\upshape\bfseries}
\setcounter{secnumdepth}{3}

\sloppy % Margón túllógó sorok tiltása.
\widowpenalty=10000 \clubpenalty=10000 %A fattyú- és árvasorok elkerülése
\def\hyph{-\penalty0\hskip0pt\relax} % Kötőjeles szavak elválasztásának engedélyezése


%--------------------------------------------------------------------------------------
% Setup hyperref package
%--------------------------------------------------------------------------------------
\hypersetup{
    % bookmarks=true,            % show bookmarks bar?
    unicode=true,              % non-Latin characters in Acrobat's bookmarks
    pdftitle={\vikcim},        % title
    pdfauthor={\szerzoMeta},    % author
    pdfsubject={\vikdoktipus}, % subject of the document
    pdfcreator={\szerzoMeta},   % creator of the document
    pdfproducer={},    % producer of the document
    pdfkeywords={},    % list of keywords (separate then by comma)
    pdfnewwindow=true,         % links in new window
    colorlinks=true,           % false: boxed links; true: colored links
    linkcolor=black,           % color of internal links
    citecolor=black,           % color of links to bibliography
    filecolor=black,           % color of file links
    urlcolor=black             % color of external links
}


%--------------------------------------------------------------------------------------
% Set up listings
%--------------------------------------------------------------------------------------
\definecolor{lightgray}{rgb}{0.95,0.95,0.95}
\lstset{
	basicstyle=\scriptsize\ttfamily, % print whole listing small
	keywordstyle=\color{black}\bfseries, % bold black keywords
	identifierstyle=, % nothing happens
	% default behavior: comments in italic, to change use
	% commentstyle=\color{green}, % for e.g. green comments
	stringstyle=\scriptsize,
	showstringspaces=false, % no special string spaces
	aboveskip=0.5em,
	belowskip=0.5em,
	backgroundcolor=\color{lightgray},
	columns=flexible,
	keepspaces=true,
	escapeinside={(*@}{@*)},
	captionpos=b,
	breaklines=true,
	frame=single,
	float=!ht,
	tabsize=2,
	literate=*
		{á}{{\'a}}1	{é}{{\'e}}1	{í}{{\'i}}1	{ó}{{\'o}}1	{ö}{{\"o}}1	{ő}{{\H{o}}}1	{ú}{{\'u}}1	{ü}{{\"u}}1	{ű}{{\H{u}}}1
		{Á}{{\'A}}1	{É}{{\'E}}1	{Í}{{\'I}}1	{Ó}{{\'O}}1	{Ö}{{\"O}}1	{Ő}{{\H{O}}}1	{Ú}{{\'U}}1	{Ü}{{\"U}}1	{Ű}{{\H{U}}}1
}


%--------------------------------------------------------------------------------------
% Set up theorem-like environments
%--------------------------------------------------------------------------------------
% Using ntheorem package -- see http://www.math.washington.edu/tex-archive/macros/latex/contrib/ntheorem/ntheorem.pdf

\theoremstyle{plain}
\theoremseparator{.}
\newtheorem{example}{\pelda}

\theoremseparator{.}
%\theoremprework{\bigskip\hrule\medskip}
%\theorempostwork{\hrule\bigskip}
\theorembodyfont{\upshape}
\theoremsymbol{{\large \ensuremath{\centerdot}}}
\newtheorem{definition}{\definicio}

\theoremseparator{.}
%\theoremprework{\bigskip\hrule\medskip}
%\theorempostwork{\hrule\bigskip}
\newtheorem{theorem}{\tetel}


%--------------------------------------------------------------------------------------
% Some new commands and declarations
%--------------------------------------------------------------------------------------
\newcommand{\code}[1]{{\upshape\ttfamily\scriptsize\indent #1}}
\newcommand{\doi}[1]{DOI: \href{http://dx.doi.org/\detokenize{#1}}{\raggedright{\texttt{\detokenize{#1}}}}} % A hivatkozások közt így könnyebb DOI-t megadni.

\DeclareMathOperator*{\argmax}{arg\,max}
%\DeclareMathOperator*[1]{\floor}{arg\,max}
\DeclareMathOperator{\sign}{sgn}
\DeclareMathOperator{\rot}{rot}


%--------------------------------------------------------------------------------------
% Setup captions
%--------------------------------------------------------------------------------------
\captionsetup[figure]{aboveskip=10pt}

\renewcommand{\captionlabelfont}{\bf}
%\renewcommand{\captionfont}{\footnotesize\it}

%--------------------------------------------------------------------------------------
% Hyphenation exceptions
%--------------------------------------------------------------------------------------
\hyphenation{Shakes-peare Mar-seilles ár-víz-tű-rő tü-kör-fú-ró-gép}


\author{\vikszerzo}
\title{\viktitle}


\begin{document}

\pagenumbering{gobble}
\selectthesislanguage
\hypersetup{pageanchor=false}
%--------------------------------------------------------------------------------------
%	The title page
%--------------------------------------------------------------------------------------
\begin{titlepage}
\begin{center}
\includegraphics[width=60mm,keepaspectratio]{figures/bme_logo.pdf}\\
\vspace{0.3cm}
\textbf{\bme}\\
\textmd{\vik}\\
\textmd{\viktanszek}\\[5cm]

\vspace{0.4cm}
{\huge \bfseries \vikcim}\\[0.8cm]
\vspace{0.5cm}
\textsc{\Large \vikdoktipus}\\[4cm]

{
	\renewcommand{\arraystretch}{0.85}
	\begin{tabular}{cc}
	 \makebox[7cm]{\emph{\keszitette}} & \makebox[7cm]{\emph{\konzulens}} \\ \noalign{\smallskip}
	 \makebox[7cm]{\szerzo} & \makebox[7cm]{\vikkonzulensA} \\
	  & \makebox[7cm]{\vikkonzulensB} \\
	  & \makebox[7cm]{\vikkonzulensC} \\
	\end{tabular}
}

\vfill
{\large December 10, 2020}
\end{center}
\end{titlepage}
\hypersetup{pageanchor=false}


\tableofcontents\cleardoublepage
\include{include/declaration}

% --- ABSTRACT --------------------------------------------------------------- %

\pagenumbering{roman}
\setcounter{page}{1}

\selecthungarian
\chapter*{Kivonat}\addcontentsline{toc}{chapter}{Kivonat}
A szakdolgozat egy modellek validációjával foglalkozó eszköz teljesítményével
kapcsolatos problémáinak nyomozásáról és orvoslásáról szól.

A modellek validációja önmagában egy erőforrásigényes feladat, így nem
meglepő, hogy egy ilyen eszközben találunk javítani valót a teljesítmény terén.
A vizsgált eszköz a VIATRA Query, melyben mintákat definiálhatunk, és
illeszthetünk modellekre az esetleges hibák felderítésére, egy kifejezetten erre
szolgáló nyelv segítségével (VIATRA Query Language, VQL). A közelmúltban több
felhasználónak is feltűnt, hogy a VQL szerkesztő indokolatlanul lassan reagál
a lekérdezések szerkesztésére, mely megakadályozza a gördülékeny munkamenetet.

A dolgozatban nagy vonalakban ismertetem a VIATRA Query-t, mint szoftvert,
valamint mutatok néhány egyszerű példát a használatára minták írásán keresztül.

A szoftver bemutatása után egy ismerten lassú példafájl használatával bemutatom
a probléma forrásának felderítését egy profilozó szoftver segítségével. Emellett
részletekbe menően ismertetem a releváns részek működését, valamint kidolgozok
egy potenciális javítást is.

Végül miután elkészült a javítás, utolsó lépésként bemutatom, hogyan lehet a
változtatásokat elküldeni a VIATRA fejlesztőinek, hogy a szoftver következő
kiadása már tartalmazhassa őket.
\vfill

\selectenglish
\chapter*{Abstract}\addcontentsline{toc}{chapter}{Abstract}
This thesis is about investigating and solving the performance issues of a tool,
which validates models.

Model validation by itself is a resource-intensive task. As such, it's not
surprising, that we can find things to improve in such a tool.
The tool we inspect is VIATRA Query, in which we can defined patterns, and match
them on models to check them for potential errors. This is done in a
domain-specific language (VIATRA Query Language, VQL). In the past, multiple
users have noticed, that the VQL editor reacts in an unreasonably slow manner to
changes in the VQL editor, which obstructs a smooth workflow.

In the thesis, I'll broadly show the VIATRA Query software, and show the
basic usage of it through some simple patterns.

After showing the tool itself, I'll present the investigation of the problem by
using a file, that is known to cause slowdowns, and a profiler software. Along
with this, I'll show the details about how that part of the software works, and
work out a potential fix for it, too.

Finally, after the my fix is in a working state, I'll show how to send changes
to the developers of VIATRA, so my fix could be included in its next release.
\vfill

% ---------------------------------------------------------------------------- %

\cleardoublepage
\selectthesislanguage
\newcounter{romanPage}
\setcounter{romanPage}{\value{page}}
\stepcounter{romanPage}
\pagenumbering{arabic}

% --- CONTENT ---------------------------------------------------------------- %

\chapter{Introduction}
\section{Introduction to VIATRA Query}
\subsection{Querying models}
\subsection{Finding patterns}
\subsection{VIATRA Query Language (VQL)}
\subsection{Formulating constraints}
\section{Work environment}
\subsection{Eclipse Platform for VIATRA development}
\subsection{YourKit Java Profiler}

\chapter{Finding the cause of the editor slowdown}
\section{Problem description}
\section{Profiling the query language editor}
\subsection{The profiling workflow}
\subsection{Dynamic rebuild}
\subsection{The functions taking the most time}
\section{Type inference in VIATRA Query}
\subsection{Re-generating metamodels}
\subsection{Probable cause of the slowdowns}
\section{Conclusion}

\chapter{Optimizing query metamodel generation}
\section{TODO SOLUTION IDEA}
\section{Implementing the chosen solution}
\subsection{TODO SOLUTION STEPS}
\section{Comparing the optimized state with the original one}
\subsection{A first look}
\subsection{Original samples}
\subsection{Fixed samples}

\chapter{Applying changes to the main code repository}
\section{Writing documentation and tests}
\subsection{The asciidoc format}
\section{Opening a bug tracker ticket}
\section{Sending the changes to version control}
\section{Asking for code review}
\section{Integrating changes into the next VIATRA version}

% ---------------------------------------------------------------------------- %

%\listoffigures\addcontentsline{toc}{chapter}{\listfigurename}
%\listoftables\addcontentsline{toc}{chapter}{\listtablename}

\addcontentsline{toc}{chapter}{\bibname}
\nocite{*} % FIXME remove
\bibliography{bib/mybib}

\label{page:last}
\end{document}
